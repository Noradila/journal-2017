\section{Conclusion}
\label{Conclusion}

This paper introduced Multichannel Routing protocol (MCRP) which consists of a protocol for multi-channel use of Wireless Sensor Networks and an algorithm for channel selection. The paper also introduced a topology changing procedure that greedily changes the topology of a WSN to improve the estimated network lifetime.  MCRP was tested in hardware for small networks, in emulation for larger networks and in simulation for networks with several hundred nodes.  MCRP shows high packet reception rate of nearly 100\% in simulations and 80\%-90\% in hardware results. In addition, MCRP reduces the energy consumption by an average of 3 times during communications as the effect of multichannel.  The energy-based tree reconfiguration is proposed to further improve the multichannel network by considering the energy level of each sensor. 
It enables the network to be fully functional for a longer period of time by extending the period until the first sensor runs out of energy.  The results showed an increase in the network lifetime by 8.3 times more for the improved tree compared to the intitial topology in a simulated 500 node system.
