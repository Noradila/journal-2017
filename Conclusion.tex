\section{Conclusion}
\label{Conclusion}

This paper introduced Multichannel Routing protocol (MCRP) which consists of a protocol for multi-channel use of Wireless Sensor Networks and an algorithm for channel selection. The paper also introduced a topology changing procedure that greedily changes the topology of a WSN to improve the estimated network lifetime.  MCRP was tested in hardware for small networks, in emulation for larger networks and in simulation for networks with several hundred nodes.  MCRP shows high packet reception rate of nearly 100\% of the ideal (zero interference) rate in simulations and 80\%-90\% in real-world tests on hardware. In addition, MCRP reduces the energy consumption at an average node by a factor of 3 due to the multichannel system avoiding interference.  The energy-based tree reconfiguration is proposed to further improve the multichannel network by considering the energy level of each sensor and forming a topology that places less load on nodes with low power. 
This enables the network to be fully functional for a longer period of time and extends sensor lifetime.  The system showed an increase in the network lifetime increasing the time until first node failure by a factor of 9 (over a static topology) in a simulated 500 node system.
