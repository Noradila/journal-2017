\section{Energy-based Tree Reconfiguration}
\label{OptimalTree}

The expected number of transmissions (ETX) is calculated by nodes in RPL.  In MCRP nodes are selected to minimise this and channels are changed to minimise interference. However this does not account for the remaining battery life of nodes.  To do this MCRP is extended to reconfigure tree partners to maximise network lifetime.  

\subsection{Details of Tree Reconfiguration}

The next step in the process is to rearrange the tree to maximise network lifetime.  In this paper we take the definition that network lifetime is the time from the network set up until the first node failure.  Therefore, maximising network lifetime can be considered to be the problem of maximising the minimum predicted node lifetime over all nodes.  We alter the topology to achieve this since node lifetime is a function of the number of messages it sends.  However arbitrary topology rearrangement is not possible since only some nodes are within communication range of others.  Instead we consider "swaps" where one node changes its parent to another node. 
We limit the number of each node possible changes to swaps, which reduces the possible combinations and makes the problem local and tractable.
%The set of potential swaps is not large since not every node can see every other node.  Hence each node only has a small number of possible parents.  

%It can be trivially shown that 
Since arbitrary changes to the topology can be realised as a series of swaps, we shall be focusing on swaps, such that: we limit complexity, and maintain locality and quick updating.
Any topology can be reached by a series of swaps where 
a single node changes its parent node (since a topology can be completely defined in terms of the parent node for each node this takes at most $n$ swaps where $n$ is the number of nodes with a different parent in the target topology).  It is useful to define the "children" of a node $i$ as those which have node $i$ as their parent and the "descendents" of node $i$ as all those nodes that are below $i$ in the tree -- that is the children of $i$, the children of those children and so on.  It should also be noted that swapping nodes in the tree that are not $i$ or descendents of $i$ can not reduce the power consumption of node $i$ (since the number of messages going through $i$ cannot be reduced).  So only swaps to $i$ or its descendents need be considered.

 %(a) swap the parent of node $i$, (b) swap the children of node $i$, and (c) swap the descendants of node $i$ that are not the children. However, swapping the parent of the minimum lifetime node does not improve the node lifetime as the number of children and descendants remain same. Thus, only option (b) and (c) are further investigated.  

We next need an estimate of node lifetime.  Let $l_i$ be the current estimated lifetime of node $i$ (in arbitrary units).  This will be a function of the current energy and the number of messages it must send and receive.  Let $e_i$ represent the amount of energy remaining in the node (as a percentage of the total energy).  Let $d_i$ be the number of descendants, $t_{ij}$ represents the number of transmissions on average required to send a message from node $i$ to node $j$ (the ETX).  Let $p(i)$ be the node which is parent of $i$ and $c(i)$ be the set of children.  
We make the simplifying assumption that each node generates messages destined for the LPBR at the same average rate and since this rate constant $R$ would appear in all equations it may be dropped.  Node $i$ will receive $d_i$ messages from its descendents and generate 1 message itself. These messages require $t_{ip(i)}$ retransmissions giving $(d_i + 1)t_{ip(i)}$ messages and retransmissions to send them to the parent.  Each $j$ in the child set $c(i)$ will similarly wish to send $d_j + 1$ messages to $i$ and these messages are repeated $t_{ji}$ times on average.  Assuming sending and receiving messages gives an equal drain on battery life we therefore have
\begin{equation}
l_i = \frac{e_i}{{(d_i + 1)t_{ip(i)} + \sum_{j \in c(i)} (d_j + 1)t_{ji}}},
\label{optimalEq}
\end{equation}
no normalising constant is needed for the rate $R$ and for the cost of receiving/sending a message since $l_i$ is in arbitrary units.  If receiving a message has a different cost than sending a message then a normalising constant can be introduced before the sum.

A greedy algorithm is used to work out which swaps should be made in order.  This algorithm works as follows:
\begin{enumerate}
\item Calculate $l_i$ for all nodes.
\item Find $i$ such that $l_i$ is minimal.  The node $i$ is the node that will run out of energy first and $l_i$ is the current network lifetime estimate.
\item Create a set of possible swaps $S$ consisting of 
\begin{enumerate}
\item Swaps that move a non-child descendent of $i$ to be a direct child of $i$ (which may reduce the cost of messages to $i$ if that node now has reduced ETX).
\item Swaps that move a descendent of $i$ (including direct children) so that it is no longer a descendent of $i$ (which will reduce the number of messages through $i$.
\item Swaps that move a child of $i$ to be a non-child descendent of $i$ (if the new route has reduced ETX).
\item Swaps that move the parent of $i$ so that $i$ has reduced ETX to its parent.
\end{enumerate}
\item From the set $S$ calculate the swap that most improves network lifetime estimate.
\item If this new network lifetime is equal or larger to the current network lifetime estimate terminate the algorithm
\item Swap to the new topology and go to step 1).
\end{enumerate}

This algorithm greedily chooses topologies that improve lifetime at each step until no more swaps are available that will improve the network lifetime.

%\begin{algorithm}
%\caption{Pseudo-code for MCRP optimal tree algorithm}
%\label{mcrp_algo}
%\begin{algorithmic}[]
%\\\textbf{Notations}
%%\\$e_i$ is the node battery power
%\\$l_i$ is the node lifetime
%\\$c_i$ is the number of node $i$ children
%\\$d_i$ is the number of node $i$ descendants
%\\\textbf{Pseudo-code}
%\\Form tree based on MCRP
%\\Update battery level for all nodes
%\\Update all nodes $l_i$, $c_i$, $d_i$
%\\minimum $\leftarrow$ 0
%\\previousSwapNode $\leftarrow$ 0 
  %\While{node $\neq$ previousSwapNode}
    %\State Find node with minimum $l_i$
    %\State List all potential $c_i$ and $d_i$ swap
	%\If{$c_i$ and $d_i$ swap $l_i$ $>$ minimum} %to avoid cycle
		%\State Recalculate all nodes $l_i$
		%\If{all new nodes $l_i$ $>$ minimum}
			%\State Update tree
			%\State New tree is optimal
		%\Else
			%\State Revert to previous optimal tree
		%\EndIf
			%\State previousSwapNode $\leftarrow$ node
	%\Else
		%\State Current tree is optimal
	%\EndIf
  %\EndWhile
%\end{algorithmic}
%\end{algorithm}

%Algorithm \ref{mcrp_algo} describes the swapping processes based on the nodes lifetime calculated from Equation \ref{optimalEq}. It considers all available paths between the nodes and shows all potential topologies before deciding on the optimal tree.
%It is assumed that all nodes residual energy and the paths are known. Both the nodes battery and the link conditions can deteriorate over time. However, it is assumed that the current selected paths are the favourable routes selected by the MCRP, thus, only the battery level of the nodes is the variable. The topology is changed accordingly where the nodes that have the minimum value is selected to balance the network in term of the battery, link conditions and the number of children and descendants. The network is considered as balanced in term of the lifetime which means, the number of nodes and descendants connected might not be fairly distributed as the battery level vary in each node.

%\subsection{Illustrative Example}

%Figure \ref{fig:ot} is an illustrative example to explain the algorithm proposed. Assumed that the tree formed in Figure \ref{fig:ot-1} is the current optimal tree after running MCRP processes. Each node is labelled with the battery level, represented in percentage for simplicity. It can also be represented in volts or Joules. The lines between the nodes represent routes in different channels where dotted lines are the potential routes and the solid lines are the current routes. The values represent the link conditions in terms of the number of successful expected transmission between the two nodes. The values of the links are the expected transmission taken only for the upwards route as the links downwards could have different values due to the different transmission and reception channels on each node thus different link quality. 
%The transmission and reception channels of a node cannot be the same to avoid interference with nearby nodes.

%The figure shows that node 2 has the most descendants which consequently reduce the node lifetime as it has to forward more packets than any other nodes. Initially, the topology is formed based on the least value on the paths. In order to optimise the tree, the overall network lifetime is considered where paths that are not the minimum could be chosen as the route as it prolongs the overall functionality of the network. In this example, node 2 has the minimum lifetime. It can be maximised through swaps. 

%There are several potential swaps to improve node 2 lifetime that includes both the children which are node 5 and 6, and the children of children, node 7 and 8. Figure \ref{fig:ot-2} shows node 5 swaps to node 4 instead of its initial node 2 and the network lifetime is calculated. Node 2 lifetime is improved, however, node 1 has a lower lifetime than the initial minimum value as the result of swapping. Node 1 now has 5 descendants while node 2 only has one when it initially had 4. In order to reduce the number of unnecessary swap, once the maximum minimum lifetime is found, all nodes lifetime values are checked to ensure that they have higher lifetime than the initial minimum lifetime regardless of the maximising the minimum node to avoid endless cycle of swaps. While the swap done by node 5 improves node 2 lifetime, node 1 lifetime deteriorates to a value lower than the minimum. The network reverts to the previous topology that is better than the new swap. Node 5 tries and swaps to node 6, then node 6 swaps to node 8. However, the potential topology is not improved. Node 2 then swaps its descendants node 7 and 8.

%When node 7 is swaps to node 4 instead of node 5, the tree is improved. It can be seen in Figure \ref{fig:ot-3} that the tree is more balanced and node 2 lifetime is prolonged. As the result of swapping, node 4 lifetime is reduced as the path from node 7 to node 4 is not the smallest path value. The tree is updated as the current optimal tree. It is not yet the final optimal tree because node 8, which is another node 2 descendant has not been checked. If node 8 swap does not improve the tree, the swap from node 7 is chosen as the final optimal tree. 
%Another potential swap is shown in Figure \ref{fig:ot-4} where node 8 is connected to node 6 instead of node 5. In both cases, node 2 lifetime is maximised and all nodes lifetime are above the minimum value. The tree in Figure \ref{fig:ot-3} is selected as the final optimal tree in maximising node 2 lifetime. Further investigations are required in order to decide the criteria on an optimal tree when there are several good topologies to be selected. 

%Node 1 is then selected as the minimum lifetime as node 2 cannot be selected again to avoid unnecessary repetition. Optimal tree from the potential swaps for node 1 is not found thus the tree is said to be optimal. In the algorithm, the same node cannot swap again right after its previous swap. This is done to avoid oscillation which would produce similar result. The node however, could swaps in the next round as the other node swap would have changed the topology.

%The swaps are assumed to happen once until the network stops functioning, thus the overheads are negligible. The swapping calculations and decisions are made by the LPBR due to sensors limitations and constraints. LPBR informs the specific nodes of the final swapping if it needs to take place. In term of energy cost, the cost is negligible as the swaps are infrequent and being controlled by the LPBR.

%\begin{figure}
%\centering
%\subfigure[Initial tree]{\label{fig:ot-1}\includegraphics[page=1, trim=2cm 9cm 2cm 2cm, clip=true, totalheight=0.19\textheight]
%{figures/ot1.pdf}}        
%%\hfill        
%\subfigure[Node 5 swaps to node 4]{\label{fig:ot-2}\includegraphics[page=2, trim=2cm 9cm 2cm 2cm, clip=true, totalheight=0.19\textheight]
%{figures/ot1.pdf}}
%%\hfill        
%\subfigure[Node 7 swaps to node 4]{\label{fig:ot-3}\includegraphics[page=3, trim=2cm 9cm 2cm 2cm, clip=true, totalheight=0.19\textheight]
%{figures/ot1.pdf}}
%%\hfill        
%\subfigure[Node 8 swaps to node 6]{\label{fig:ot-4}\includegraphics[page=4, trim=2cm 9cm 2cm 2cm, clip=true, totalheight=0.19\textheight]
%{figures/ot1.pdf}}
%\caption{Graph of the bidirectional paths in a WSN}
%\label{fig:ot}
%\end{figure}
