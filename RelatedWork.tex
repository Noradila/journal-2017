\section{Related Work}
\label{RelatedWork}

\mypar{Network Lifetime} A major aim of MCRP is to increase network lifetime by optimising energy use. Various definitions of network lifetime are used in the literature, for example, Hellman et al. \cite{hellmanLifetime} and Xu et. al. \cite{gafLifetime} defined the network lifetime as the fraction of surviving nodes in the network, while Tian and Georganas \cite{tian2002coverage} measured the time until all nodes have been drained of their energy.
%for example the mean life time of all nodes \added{[r]}. 
In this paper we use the common definition of the time until the first node runs out of energy \cite{maxmin, erapl, lifetimedef1}.  Typically in networks with a central node providing upload to the network (LBPR) nodes closer to this in the topology carry more traffic and use more energy.  Energy use also increases in the presence of interference since packets may need to be transmitted multiple times to get a successful transmission. Two main ways to extend network lifetime are (i) multichannel MAC protocol to reduce multiple transmission due to interference and (ii) reconfiguring topologies to make less demand on nodes with low energy.

\mypar{Multichannel MAC Protocols}
Existing multichannel MAC protocols can be categorised into synchronous and asynchronous. Synchronous systems require tight time synchronisation to schedule communications and avoid collisions. Asynchronous systems self-configure but avoid the costs of synchronisation.
Channel hopping enables the packet to be communicated on a different frequencies to work around interference.  Orchestra \cite{orchestra} is a synchronous protocol that is based on the Time Slotted Channel Hopping (TSCH) \cite{tsch}. It uses channel hopping to increase the reliability in the network.  MiCMAC \cite{micmac} is an asynchronous MAC protocol that uses a distributed channel hopping protocol and switches to a different channel each time it wakes up.  Orchestra and MiCMAC use predefined hopping sequences in the hope of routing around channels with interference.  They use a limited subset of channels known to have low interference in typical settings. Previous work by Sha et al. \cite{homearea} and Mohammad et al. \cite{oppcast} found that the channel reliability changes over time in non cyclic manner, thus no specific channels could achieve a long term reliability. Stabellini \cite{energyluca} proposed a spectrum sensing algorithm to decide on the number of channels to be sensed before the channel is selected for transmissions.  By contrast to these approaches MCRP attempts to detect which channels have interference during operation and avoid them as the nodes connect.

\mypar{Routing Protocols}  
A major issue in WSN routing protocols is finding and maintaining energy efficient routes.  Topologies may need to change rapidly to avoid network disconnection due to interference.  However, topology changes can be costly in terms of messaging and packet loss.  RPL is a routing protocol that builds the topology based on an objective function related to link quality.  New metrics and constraints can be defined for RPL and this leads to several studies combining energy based metrics into RPL objectives~\cite{energyrpl,energyLHC,elt,customOF,roee,compositeMetric,caof}.  Other studies increase network lifetime by distributing communication load in the network as proposed by Liu et al. \cite{loadbalance} and Delaney et al. \cite{spreadload} to avoid overusing nodes.  The studies in ELT~\cite{elt}, neighbourhood metrics routing~\cite{spreadload} and LB-RPL~\cite{loadbalance} move workloads to avoid overloading individual nodes.  Like MCRP these often use expected number of transmissions (ETX) as a metric. Other metrics such as the location and resource oriented were also considered to increase the efficiency of the nodes.
Like MCRP,  ELT~\cite{elt} aims to maximise the minimum node lifetime while~\cite{energyrpl} and ROEE \cite{roee} aim to minimise the maximum residual energy. Neighbourhood metrics routing~\cite{spreadload}, energy-oriented routing~\cite{loadbalance}, ELT~\cite{elt}, $L^{2}AM$~\cite{compositeMetric,customOF} aim to have a network whose nodes deplete at similar speed.  
 
