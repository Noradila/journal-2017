\section{Introduction}

% The very first letter is a 2 line initial drop letter followed
% by the rest of the first word in caps.
% 
% form to use if the first word consists of a single letter:
% \IEEEPARstart{A}{demo} file is ....
% 
% form to use if you need the single drop letter followed by
% normal text (unknown if ever used by the IEEE):
% \IEEEPARstart{A}{}demo file is ....
% 
% Some journals put the first two words in caps:
% \IEEEPARstart{T}{his demo} file is ....
% 
% Here we have the typical use of a "T" for an initial drop letter
% and "HIS" in caps to complete the first word.
%\IEEEPARstart{T}{his} demo file is intended to serve as a ``starter file''
%for IEEE journal papers produced under \LaTeX\ using
%IEEEtran.cls version 1.8b and later.
% You must have at least 2 lines in the paragraph with the drop letter
% (should never be an issue)

\IEEEPARstart{W}{ireless} sensor networks (WSNs) are widely used to gather data and measurement \added{from the physical world in order to enable advanced monitoring and control of engineered or physical infrastructures [r]. In the last few years, a concerted effort has begun to make WSNs ``internet-enabled'', in order to allow for advanced monitoring or reaction capabilities that cannot be achieved by local processing alone. Within this Internet-of-Things (IoT) paradigm,}  it is increasingly important to have reliable and energy-efficient WSNs that could function for years, as sensors are often deployed in areas that are difficult to reach, such as volcanic monitoring [r], forest fire detection [r], monitoring of dam structural integrity [r], etc.

However, \added{sensors have limited energy resources} as they are battery powered, and will not able to function once a certain threshold of energy level is reached. 
\added{IoT-oriented WSNs also operate in unlicensed bands, such as the 16 channels of the 2.400-2.484 GHz band [r] or the 7 channels of the 5.850-5.925 GHz band [r], where intereference from the unreliable radio environment} drains the sensors' batteries at a higher rate. These constraints have a major impact on the \added{WSN's throughput and lifetime characteristics. In order to cope with these limitations, one}  should be able to \added{adaptively  switch to reliable channels in order to reduce the effect of interference in the throughput and and energy drain of each sensor, and thereby maximize the} overall network lifetime.
Many energy-efficient \added{medium access control (MAC) or routing protocols have been proposed [r][r], with simulations showing promising results in advancing either throughput or energy autonomy, but very few joint MAC\ \&\ routing protocol for throughput and lifetime maximization have been experimentally tested in unlicensed bands yet}. \textcolor{red}{SEE\ NOTE\ 1}.

Multichannel cross-layer routing protocol (MCRP) is a new protocol that considers all available \added{channels in the unlicensed spectrum}. The protocol is partly distributed: nodes work independently, but strategic decisions are made by a centralised controller known as the low power border router (LPBR), \added{which is also the WSN aggregator that communicates with internet resources within an IoT-oriented deployment}. MCRP increases throughput by using the channels that are found to have the lowest interference for communications.
Energy-based tree reconfiguration is \added{then} used to improve network lifetime.  We show that MCRP is able to achieve 80\%-90\% of the optimal \added{(i.e., interference-free)} throughput in real world experiments.
MCRP also improves the energy efficiency consuming three times less energy than a single channel protocol.  It allows reconfiguration of topologies to increase the network lifetime, as measured by time to first node failure.  In simulation experiments, the time to first node failure was increased by a factor of 8.3 \added{in comparison to the no-reconfiguration case}.

An initial version of MCRP was introduced \added{in our related conference paper}\cite{mcrp}, which presented some basic emulation results for loss rates on a small network and a comparison with the single channel RPL [r] \textcolor{red}{DEFINE\ ACRONYM\ AND\ ADD\ REFERENCES\ TO\ ALL\ [r] MENTIONS}.  This paper extends the MCRP protocol to improve recovery and to allow topology reconfiguration.  It presents more extensive emulation results as well as real world implementation on hardware with a ten node network and simulation results on a several hundred node network.  The results are compared with another multi-channel protocol, Orchestra [r], \added{which represents the state-of-the-art}.

This paper is organised as follows. Section \ref{RelatedWork} presents  related work, Section \ref{MCRP} explains the multichannel cross-layer routing protocol, the improvements in comparison to a single and existing multichannel protocols and evaluates MCRP performance in emulation and real world environments.
Section \ref{OptimalTree} describes the proposed energy-based tree reconfiguration for WSNs in detail.  Section \ref{MCRPemulation} tests MCRP in emulation \textcolor{red}{EMULATION OR SIMULATION?? THE TWO ARE NOT THE SAME THING. E.G., COOJA IS AN EMULATOR WHILE NS2 IS A SIMULATOR...} and Section \ref{MCRPhardware} tests it in hardware.
Section \ref{PerformanceEvaluation} evaluates the performance of the proposed optimisation. Finally, Section \ref{Conclusion} concludes this paper.
