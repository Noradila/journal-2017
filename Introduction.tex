\section{Introduction}

% The very first letter is a 2 line initial drop letter followed
% by the rest of the first word in caps.
% 
% form to use if the first word consists of a single letter:
% \IEEEPARstart{A}{demo} file is ....
% 
% form to use if you need the single drop letter followed by
% normal text (unknown if ever used by the IEEE):
% \IEEEPARstart{A}{}demo file is ....
% 
% Some journals put the first two words in caps:
% \IEEEPARstart{T}{his demo} file is ....
% 
% Here we have the typical use of a "T" for an initial drop letter
% and "HIS" in caps to complete the first word.
%\IEEEPARstart{T}{his} demo file is intended to serve as a ``starter file''
%for IEEE journal papers produced under \LaTeX\ using
%IEEEtran.cls version 1.8b and later.
% You must have at least 2 lines in the paragraph with the drop letter
% (should never be an issue)

\IEEEPARstart{W}{ireless} sensor networks (WSNs) are widely used in various kinds of applications to collect data and measurements data from the sensors. 
The sensors are mainly deployed to track and monitor in different types of environments such as on the land, underground and underwater for continuous sensing, event detection, location sensing and other control over the different components of the sensing device. 
It is increasingly important to have reliable and energy efficient WSNs that could function for years as sensors are easily deployed in areas that are difficult to reach such as for volcanic monitoring, forest fire detection and flood detection.

However, sensors suffer from limited hardware resources which only allow limited computational functionalities to be performed. Sensors also suffer from limited energy capacities as they are battery powered and will not able to function once the certain threshold of energy level is reached. 
Sensors also operate in an unreliable radio environment that is noisy and error prone which drain the sensors batteries at a higher rate. These constraints have a major impact on the sensors performance. 
In order to cope with the sensors limitations, the protocol should be able to use many reliably channels for communications to reduce the effect of interference and energy consumption, and to have an optimal energy-based routing to prolong the sensors and overall network lifetime.
Many energy efficient protocols have been proposed in term of the Medium Access Control (MAC) protocols and routing protocols that show promising results in overcoming the problems but none is widely implemented yet.

Multichannel cross-layer routing protocol (MCRP) is a protocol that considers all available channels in the spectrum. The protocol is partly distributed: nodes work independently with strategic decisions made by a centralised controller known as the low power border router (LPBR). MCRP increases throughput by using low interference channels for communications.
Energy-based tree reconfiguration is used to improve network lifetime.  MCRP attempts to assess interference on a channel and reconfigure to use channels with less interference.
We show that MCRP also shows high throughput values of 80\%-90\% of the optimal (zero interference) throughput using real world experiments.
MCRP also improves the energy efficiency consuming 3 times less energy than a single channel protocol.  It allows reconfiguration of topologies to increase network lifetime as measured by time to first node failure.  In the experiments, the network lifetimes increased by 8.3 times.

An initial version of MCRP was introduced in~\cite{mcrp} which presented some basic emulation results for loss rates on a small network and comparison with the single channel RPL.  This paper extends the MCRP protocol to improve recovery and to allow topology reconfiguration.  It presents more extensive emulation results as well as real world implementation on hardware with a ten node network and simulation results on a several hundred node network.  The results are for the first time compared with another multi-channel protocol, Orchestra.

This paper is organised as follows. Section \ref{RelatedWork} presents the related work, Section \ref{MCRP} explains the multichannel cross-layer routing protocol, the improvements in comparison to a single and existing multichannel protocols and evaluates MCRP performance in emulation and real world environments.
Section \ref{OptimalTree} describes the proposed energy-based tree reconfiguration for WSNs in detail.  Section \ref{MCRPemulation} tests MCRP in emulation and section \ref{MCRPhardware} tests it in hardware.
Section \ref{PerformanceEvaluation} evaluates the performance of the proposed optimisation. Finally, Section \ref{Conclusion} concludes this paper.
