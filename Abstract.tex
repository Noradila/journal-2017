\begin{abstract}
Internet-of-Things oriented data gathering  can be abstracted as wireless sensor networks (WSNs) that upstream measurements to aggregators in order to link to internet resources for analysis and reaction capabilities that surpass the aptitude of local processing. Within this context, the WSN part typically comprises an ad-hoc  deployment in a number of unlicensed spectral bands (channels) with unpredictable interference characteristics.  This results in frequent packet retransmissions  and high energy drain.
This paper presents a two-step technique to optimise such IoT deployments in terms of the throughput and WSN operational lifetime. 
The proposed multichannel cross-layer routing protocol (MCRP) detects excessive interference at the medium access control (MAC) layer and switches away from channels that experience excessive interference using a graph colouring approach. In order to prolong network lifetime, MCRP also incorporates a tree reconfiguration algorithm to find network topologies that use alternative routes based on the sensors' current energy status. 
Experimental results demonstrate that MCRP can achieve 80\% to 90\% of the throughput of the ideal (i.e., interference-free) MAC layer. In addition, simulated tests on large topologies show that MCRP allows \textcolor{red}{an improvement by a factor of 9 in the WSN's lifetime.}
%for more than \textcolor{red}{8 times improvement over the baseline} in the WSN's lifetime.
%increased by a factor of 8.3 in comparison to the no-reconfiguration case
%eight-fold improvement in the WSN's lifetime.
\end{abstract}
