\begin{abstract}
\added{Internet-of-Things oriented data gathering  can be abstracted as wireless sensor networks (WSNs) that upstream measurements to aggregators in order to link to internet resources for analysis and reaction capabilities that surpass the aptitude of local processing. Within this context, the WSN part typically comprises} an ad-hoc  deployment  \added{in a number of unlicensed spectral bands (channels) with unpredictable interference characteristics}.  This results in \added{frequent} packet retransmissions  and high energy drain.
This paper presents a two-step technique to optimise \added{such IoT deployments} in terms of the throughput and \added{WSN operational} lifetime. 
The \added{proposed multichannel cross-layer routing} protocol \added{(MCRP)} detects \added{excessive interference at the medium access control (MAC) layer} and \added{switches away from} channels that  \added{experience excessive} interference \added{using a graph coloring approach}. In order to prolong network lifetime, \added{MCRP also incorporates} a tree reconfiguration \added{algorithm} to find network topologies \added{that} use alternative routes based on the sensors' current energy \added{status}. 
Experimental results demonstrate that \added{MCRP can achieve 80\% to 90\%} of the throughput of \added{the ideal (i.e., interference-free) MAC layer}.
 \added{In addition, simulated tests on large topologies show that MCRP\  allows for more than eight-fold} improvement in the WSN's lifetime.
\end{abstract}
