\begin{abstract}
Wireless Sensor Networks (WSNs) are ad-hoc networks that consist of sensors that typically use low power radios to connect to the Internet. Unfortunately, the channels used by the sensors often suffer from interference from other devices that share the same frequency.  This results in packet retransmissions and losses and hence high energy drain rate to send a packet.
This paper presents a two step technique to optimise multichannel sensor network in term of the throughput and energy lifetime to prolong the network functionality period. 
A multichannel cross-layer routing protocol is proposed to alleviate the effect of interference to improve the network efficiency, reliability and throughput. 
The protocol detects and changes the channels that suffer from interference through the channel switching processes.
Real world experimental results demonstrate that the protocol can achieve 80\% of the loss performance of a perfect system with zero interference.
In order to prolong network lifetime in addition to multichannel, energy-based tree reconfiguration is proposed to find network topologies with improved energy use characteristics. This reconfiguration improves the overall network lifetime by balancing the sensors to use alternative routes based on the sensors current energy load. It shows an improvement of the sensors lifetime by 8.3 times in simulation tests on large networks.
\end{abstract}
