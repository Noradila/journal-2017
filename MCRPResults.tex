%\section{MCRP Emulation Results}
\section{MCRP Performance Evaluation}
\label{MCRPemulation}

\subsection{MCRP Emulation Results}

MCRP is evaluated in the Cooja network emulator.  
Fifteen nodes are emulated for the communications network one LPBR and fourteen forming the tree of sensors.  In addition 16 emulated nodes are used for interference.  
%TODO how are the nodes distributed.
Interference is generated using the model in~\cite{interferenceModel} using settings based on measurements in~\cite{radio2009}. The model has a clear state and an interfered state.  The rules for transitions between states are given in~\cite{interferenceModel}.  Here we allow four states according to the percentage of time where interfering signals are present: no interference, mild (75\% of the time no interference happens), moderate (50\% no interference) and extreme (25\% of the time no interference happens).  However, channel 26 is kept clear from interference in order to ensure RPL set up is unaffected. 

\begin{table}
\caption{\textcolor{red}{Table of parameters}}
\label{table1}
\begin{center}
\begin{tabular}{c c }
\hline
 Name of parameters & Description \\ 
 \hline
Type of sensors & TelosB \\
Number of sensors & 16 \\
Interference sensors & 16 \\
Interference Model & xxequation \\
Channels & 16, 26 is kept clear \\ 
\hline   
\end{tabular}
\end{center}
\end{table}

The emulation has the following phases:
\begin{enumerate}
\item 0-5 minutes -- set up phase with no interference (this prevents RPL failing to build an initial tree in high interference situations).
\item 5-30 minutes -- interference nodes are switched on.  The MCRP tree building and reconfiguration begins.
\item 30-60 minutes -- interference continues and nodes attempt to send data every 30--60 seconds.  The rate of successful transmission is measured.
\end{enumerate}
Losses are not measured in the tree building phase since that represents a set up cost and it is performance in the steady state that is important since this should last for weeks or years rather than tens of minutes.

The performance of MCRP is compared against the standard ContikiMAC with RPL and Orchestra to demonstrates MCRP abilities in dealing with external and intra interferences. 
%Orchestra does not support RPL downwards routing due to limited memory in TelosB. It however, support the upwards traffic which is require in the experiments as all traffics are directed upwards towards the LPBR.
MCRP is analysed using an end-to-end packet delivery performance metric, setup overhead, channel switching and reconnection delay in MCRP. The transmission success rate is calculated from the sender to the receiver over multiple hops. 
The simulations are repeated ten times. In all plots, the mean value of the ten simulations is plotted with error bars corresponding to one standard deviation in either deviation to give a measure of repeatability. The plots are of the proportion of received packets (from 0\% to 100\%) against time where the loss is measured over the previous time period.  The x-value is shifted slightly left and right to prevent error bars overlapping.

\subsubsection{Packet Loss Rates}

%\begin{figure}
%\centering
%\includegraphics[width=0.45\textwidth]{figures/single_channel.pdf}
%\caption{Emulation: Level of packet loss for mild, moderate and extreme interference levels using single channel}
%\label{fig:interference}
%\end{figure}

%Figure \ref{fig:interference} shows the results in emulation for the single channel RPL protocol. This can be taken as a baseline for improvement for any multi-channel system.  It can be seen that as the emulation proceeds the proportion of received packets falls off.  The network 

Packet loss emulation results for interference on a single channel RPL system can be found in~\cite{mcrp}. These show that when interference is introduced to single channel RPL then the packet loss rate climbs quickly.  In the extreme interference case packet loss had climbed to 100\% by the end of the emulated period. For moderate interference this was 90\%.  For mild interference the figure was only 20\% but the variance was high.  These should be kept in mind as the levels a multiple channel system should improve upon.

\begin{figure}
\centering
\subfigure[Fixed layout]
{\label{fig:randomInterference}\includegraphics[width=0.45\textwidth]{figures/ri.pdf}}
\subfigure[Random layout]
{\label{fig:randomAll}\includegraphics[width=0.45\textwidth]{figures/ra.pdf}}
\caption{Emulation: Level of packet loss for MCRP and Orchestra}
\label{fig:layouts}
\end{figure}


To evaluate MCRP capability to cope with multi-source interference we compare with an existing multichannel protocol Orchestra.  We emulate (i) fixed layout, random interference channels and (ii) random node layout, random interference channels.
%in an area of $150m^2$.  In the fixed layout nodes were positioned in order that the fourteen nodes formed a tree where the LPBR has two children and each of those children has two children (four grand children) and each of those has two children (eight great grand children).
In the fixed layout, nodes were positioned at least 10 metres apart so  only one tree was possible with the LPBR having two nodes directly attached and each of these child nodes having two nodes attached and each of these having two grandchild nodes. This produced a total of fifteen nodes including the LPBR.  10 metres was chosen so that nodes could form a good connection to the nodes next to them in the tree but nodes two hops away were too far to connect. In the random layout nodes were placed by choosing a uniformly random location within a square of side 150 metres.
Interference is chosen randomly for the 16 channels so that 4 experience each of the interference conditions: no interference, mild interference, medium interference and extreme interference.  In the experiments, Orchestra uses channel hopping on all 16 channels. 


Figure \ref{fig:layouts} show MCRP and Orchestra results for the fixed and random nodes layouts.  In these graphs (and all graphs in this paper, x-axis values are shifted slightly to avoid error bars overlaying).
MCRP performs extremely well in both scenarios as the average packet reception rates are between 90\%-100\% and the protocol successfully detects the channels with interference.
Orchestra has higher packet loss compared to MCRP, showing a maximum of 40\% packet loss on average as the channels with interference are being used for transmission periodically. 
In comparison, MCRP selects certain channels to change into after checking the channels condition which gives MCRP a smaller number of packet loss.
MCRP avoids the interference channel while Orchestra hops to the next channel in the next iteration for transmission.

\subsubsection{Overheads of multiple channel protocols}

Changing channels and probing channels requires signalling overhead for set up.  Extra messages need to be passed for probing and to change channels.  Estimations for setup overhead are given in ~\cite{mcrp}.  Default RPL on ContikiMAC for the topology considered in these experiments completed its set up using 276 packets. MCRP, the multi-channel protocol completed its set up in 716 packets, that is an overhead of 440 packets on top of RPL.  However, it is worth mentioning that this is a one-off cost representing (in this set up) approximately one hour of extra packets where the sensor network deployment would be expected to be in place for weeks, months or longer.  The set up time is 1154 seconds longer for MCRP than RPL.

MCRP also has overhead compared to RPL in terms of reconnection time.  MCRP nodes that disconnect must probe multiple channels to reconnect to the network (as described in section \ref{sec:switching}).  This is obviously slower than RPL that only need probe a single channel.  To test this in isolation we deliberately disconnected a network in the absence of interference by disabling a parent node and measuring the effect on packet transmission.

In the Figure~\ref{fig:reconnectionLayout} node 3 is disabled.
Node 6 and 7 route through node 3 to the LPBR. Nodes 6 and 7 must reconnect through node 5 as shown in Figure \ref{fig:r2}. 

\begin{figure}
\centering
\subfigure[Initial tree]{\label{fig:r1}\includegraphics[trim=2cm 15cm 2cm 2cm, clip=true, totalheight=0.12\textheight]
{figures/reconnection1.pdf}}        
\hfill        
\subfigure[Reconstructed tree when node 3 is undetected]{\label{fig:r2}\includegraphics[trim=2cm 15cm 2cm 2cm, clip=true, totalheight=0.12\textheight]
{figures/reconnection2.pdf}}
\caption{A simple tree to emulate the tree reconnection}
\label{fig:reconnectionLayout}
\end{figure}

Figure \ref{fig:reconnection} shows the result of disconnecting node 3 after 54 minutes of emulation.  In MCRP, it took between five and seven minutes before nodes 6 and 7 reconnect.  RPL with single channel is slightly quicker taking three to five minutes. However, considering the more complex task MCRP is performing two minutes added onto reconnection time in the context of a network with low data rates should be acceptable considering the other performance advantages.  In the Orchestra system, by contrast, reconnection is almost instantaneous since the synchronous Orchestra system checks nodes every slot but this is at the expense of greatly increased control traffic.

\begin{figure}
\centering
\includegraphics[width=0.45\textwidth]{figures/reconnect.pdf}
\caption{Emulation: Reconnection time taken for MCRP and single channel}
\label{fig:reconnection}
\end{figure}
